\section{features}
A fully interactive whitted-style raytracer with run time scene loading. Accelerated rendering using BVH traversal.

\subsection{Run time scene loader}
    \begin{enumerate}
    \item Json scene loader
        \begin{enumerate}
        \item Triangle meshes --- loaded from file, and transformed into the world (translate, rotate, scale)
        \item Point and Spot lights
        \item camera
        \end{enumerate}
    \item \verb|.obj| file loader (using tinyojb)
    \item \verb|.mtl| material file loader (using tinyobj)
    \end{enumerate}

\subsection{Materials}
    \begin{enumerate}
    \item diffuse lighting with hard shadows
    \item reflections
    \item transparency with refraction indices
    \item specular highlights
    \item diffuse, reflective and specular values are defined by a 3-value color to match the \verb|.mtl| file format
    \item they can also be combined in arbitrary ways (i.e.\ a material can have a diffuse, reflective, transparent and specular components)
    \item triangle meshes support smoothing using barycentric interpolation of the vertex normals
    \end{enumerate}

\subsection{Camera}
    \begin{enumerate}
    \item resizable windows
    \item zoom (mousewheel)
    \item pitch, yaw and translate
    \end{enumerate}

\subsection{Lights}
    \begin{enumerate}
    \item multi-color lighting
    \item point lights
    \item spot lights with linear falloff
    \end{enumerate}

\subsection{Other}
    \begin{enumerate}
    \item multi-platform with CMake (linux, mac and windows)
    \item multi-threaded with OpenMP (got around 5x speedup on linux, which is what is expected for 4 cores with HT)
    \end{enumerate}

\section{Building}

The source is built using the standard CMake envirnonment.

For example, on unix systems this command will do a release build -

\verb|cmake -DCMAKE_BUILD_TYPE=Release . ;| \\
\verb|make|

A similar command should work on Visual Studio, although the \verb|CMAKE_BUILD_TYPE| is not required - this is set in the Visual Studio gui.

On Windows, SDL2.dll must be available somewhere on the path or in the same dir as the binary. 

Boost is required for the test tree, but not the main system.

\section{Running}

There is a collection of test scenes, objects and materials in the data directory. To avoid fighting with relative paths, it works best to cd to this directory before running the system. For example, on windows, assuming the binary is in the Release directory under the root source tree - 

\verb|cd <source_root>\data| \\
\verb|..\Release\ray.exe col-lights.scene|

\section{Input Commands}
    \begin{enumerate}
    \item WASD - move around the world
    \item mouse movement - rotate camera (yaw/pitch)
    \item mouse wheel - zoom in/out (changes FOV)
    \item R - reset camera view to default (possibly set from scene file)
    \item M - Toggle barycentric smoothing
    \item P - write screenshot image
    \item C - print the current camera parameters
    \item B - change BVH construction method
    \item ESC - quit
    \item 0 - view ray-traced image (default)
    \item 1 - visualise per-pixel render time (more red = more render time)
    \item 2 - visualise surface normals (useful for debugging and verifying smoothed normals)
    \item 3 - visualise BVH leaf nodes
    \item 4 - visualise triangles checked
    \item 5 - visualise BVH splits traversed
    \item 6 - visualise BVH leaves checked 
    \item 7 - visualise BVH node index
    \end{enumerate}

\section{External Packages}
The following external packages were used 
\begin{enumerate}
    \item tiny object loader --- \url{https://syoyo.github.io/tinyobjloader/}
    \item glm --- \url{https://glm.g-truc.net/0.9.8/index.html}
    \item JSON for Modern C++ --- \url{https://github.com/nlohmann/json}
    \item SDL2 --- \url{https://www.libsdl.org/download-2.0.php}
    \item boost (test tree only) --- \url{https://www.boost.org/}
    \item CMake --- \url{https://cmake.org}
\end{enumerate}

\section{Screencaps}

%\includegraphics[width=0.5\textwidth]{img/colPointLights1sphere}
